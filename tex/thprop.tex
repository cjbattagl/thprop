% \documentclass{siamart171218}
\documentclass[12pt]{report}
% \documentclass[conference,10pt]{IEEEtran}
\usepackage{booktabs} % For formal tables
\usepackage{amssymb,amsmath}
\usepackage{amsfonts}
\usepackage[mathscr]{eucal}
\usepackage{bm}
\usepackage{braket}
\usepackage{xfrac}
\usepackage{relsize}
\usepackage{xspace}
\usepackage{xcolor}
\usepackage{graphicx}
\usepackage[caption=false]{subfig}
\usepackage{array}
\usepackage{siunitx}
\sisetup{per=slash, load=abbr}
\usepackage{amsbsy}
\usepackage{cleveref}
\usepackage{algpseudocode}
\usepackage{algorithm}
\usepackage{tikz}
\usepackage{url}
\usepackage{pgfplots}
\usepgfplotslibrary{colorbrewer,colormaps}
\pgfplotsset{compat=1.13}
\usetikzlibrary{pgfplots.groupplots}
\usepackage{pgfplotstable}
\usepackage{etoolbox}
\usepackage[normalem]{ulem}
%% Used for papers with subtables created with the subfig package
% \captionsetup[subtable]{position=bottom}
% \captionsetup[table]{position=bottom}
%% For referencing line numbers
\Crefname{ALC@unique}{Line}{Lines}
%% For creating math operators
\usepackage{amsopn}
% \graphicspath{{figs/}}
%
\Crefname{ALC@unique}{Line}{Lines} 
\DeclareMathOperator*{\argmin}{arg\,min}
\newcommand{\T}[2][]{\boldsymbol{#1\mathscr{\MakeUppercase{#2}}}}
\newcommand{\M}[2][]{{\bm{#1\mathbf{\MakeUppercase{#2}}}}} 
\newcommand{\Mn}[3][]{{\bm{#1\mathbf{\MakeUppercase{#2}}}}_{#3}} 
\newcommand{\bigO}[1]{\ensuremath{\mathcal{O}\hspace{-.2em}\left({#1}\right)}}
\newcommand{\inlineBigO}[1]{\ensuremath{\mathcal{O}({#1})}}
\newcommand{\MnC}[3]{\V{#1}^{(#2)}_{#3}} % Column of matrix in sequence
\newcommand{\V}[1]{\ensuremath{\mathbf{\lowercase{#1}}}\xspace} % vector
\newcommand{\R}{\ensuremath{\mathbb{R}}\xspace}
\newcommand{\varXcopy}{\ensuremath{\widetilde{\T{X}}}\xspace}
\newcommand{\varYcopy}{\ensuremath{\widetilde{\T{Y}}}\xspace}
\newcommand{\varXmat}{\mathVarSubMbox{\M{X}}{mat}\xspace}
\newcommand{\varYmat}{\mathVarSubMbox{\M{Y}}{mat}\xspace}
\newcommand{\varSz}{\varRm{sz}\xspace}
\newcommand{\varOrder}{\varRm{order}\xspace}
\newcommand{\varOutSz}{\varRm{out\_sz}\xspace}
\newcommand{\trans}{\ensuremath{\mathsf{T}}}
\newcommand{\codeCommentLine}[1]{\STATEx \quad {\color{gray}// #1}}
\newcommand{\Rtrue}{R_{\rm true}}
\newcommand{\cbcomment}{\textcolor[rgb]{1,0,0}{CB: }\textcolor[rgb]{1,0,1}}
\newcommand{\qtext}[1]{\quad\text{#1}\quad}
\newcommand{\red}[1]{\textcolor{red}{#1}}
\newcommand{\LineRef}[1]{\hyperref[#1]{line~\ref{#1}}}
\newcommand{\modeidx}{\ensuremath{m}}
\newcommand{\samplesize}{\ensuremath{S}}
\newcommand{\Tra}{{\sf T}} 
\newcommand{\parens}[1]{(#1)}
\newcommand{\Parens}[1]{\left(#1\right)}
\newcommand{\dsquare}[1]{\llbracket #1 \rrbracket}
\newcommand{\Dsquare}[1]{\left\llbracket #1 \right\rrbracket}
\newcommand{\curly}[1]{\{ #1 \}}
\newcommand{\Curly}[1]{\left\{ #1 \right\}}
\newcommand{\Real}{\mathbb{R}}
\newcommand{\datafile}{}
%
\newcommand{\TODO}{\red}
\newcommand{\GB}[1]{\textcolor{blue}{#1}}
\newcommand{\CB}[1]{\textcolor{green}{#1}}
\newcommand{\VE}[3][]{#1{\MakeLowercase{#2}}_{#3}} 
\newcommand{\Vn}[3][]{{\bm{#1\mathbf{\MakeLowercase{#2}}}}^{(#3)}} 
\newcommand{\VnTra}[3][]{{\bm{#1\mathbf{\MakeLowercase{#2}}}}^{(#3)\Tra}} 
\newcommand{\VnE}[4][]{#1{\MakeLowercase{#2}}^{(#3)}_{#4}} 
\newcommand{\MnTra}[3][]{{\bm{#1\mathbf{\MakeUppercase{#2}}}}^{(#3)\Tra}} 
\newcommand{\MC}[3][]{\V[#1]{#2}_{#3}} 
\newcommand{\MnCTra}[4][]{\VnTra[#1]{#2}{#3}_{#4}} 
\newcommand{\ME}[3][]{#1{\MakeLowercase{#2}}_{#3}} 
\newcommand{\MnE}[4][]{#1{\MakeLowercase{#2}}^{(#3)}_{#4}} 
\newcommand{\TS}[3][]{\M[#1]{#2}_{#3}}
\newcommand{\TE}[3][]{#1{\MakeLowercase{#2}}_{#3}}
\newcommand{\Mz}[3][]{\M[#1]{#2}_{(#3)}}
\newcommand{\Oprod}{\circ} 
\newcommand{\Kron}{\otimes} 
\newcommand{\Khat}{\odot} 
\newcommand{\Hada}{\ast} 
\newcommand{\BigHada}{\mathop{\mbox{\fontsize{18}{19}\selectfont $\circledast$}}} 
\newcommand{\Divi}{\varoslash}
\newcommand{\Tsize}[2]{\ensuremath#1_1 \times #1_2 \times \cdots \times #1_#2}
\newcommand{\Tidx}[2]{\ensuremath#1_1 #1_2 \cdots #1_#2}
\newcommand{\Tentry}[2]{\ensuremath(#1_1,#1_2,\dots,#1_#2)}
\newcommand{\nrank}[1][n]{\ensuremath\text{rank}_#1}
\newcommand{\trank}{\ensuremath{r_t}}
%
\newcommand{\hosvd}{HOSVD\xspace}
\newcommand{\sthosvd}{ST-\hosvd}
\newcommand{\thosvd}{T-\hosvd}
\newcommand{\hosvdsketch}{\hosvd-Sketch\xspace}
\newcommand{\dTTM}{dc-TTM}
\newcommand{\MTFSBC}{MTFSBC\xspace}

\newtheorem{theorem}{Theorem}[section]
\newtheorem{corollary}{Corollary}[theorem]
\newtheorem{lemma}[theorem]{Lemma}

\theoremstyle{definition}
\newtheorem{definition}{Definition}[section]
 
\theoremstyle{remark}
\newtheorem*{remark}{Remark}


%
\DeclareMathOperator{\EX}{\mathbb{E}}
\definecolor{col1}{HTML}{a6611a}
\definecolor{col2}{HTML}{dfc27d}
\definecolor{col3}{HTML}{80cdc1}
\definecolor{col4}{HTML}{018571}

\definecolor{cpalsncolor}{HTML}{A6611A}
\definecolor{cpalsrcolor}{HTML}{DFC27D}
\definecolor{cprandcolor}{HTML}{80CDC1}
\definecolor{cprandfcolor}{HTML}{018571}
% \pgfplotscreateplotcyclelist{barlist}{%
%   {1 of Dark2-4, fill=1 of Dark2-4},
%   {2 of Dark2-4, fill=2 of Dark2-4},
%   {3 of Dark2-4, fill=3 of Dark2-4},
%   {4 of Dark2-4, fill=4 of Dark2-4}
% }
\pgfplotscreateplotcyclelist{barlist}{%
  {col1, fill=col1},
  {col2, fill=col2},
  {col3, fill=col3},
  {col4, fill=col4}
}

% \pgfplotscreateplotcyclelist{barlist}{%
%   {ACMBlue, fill=ACMBlue},
%   {ACMPurple, fill=ACMPurple},
%   {ACMGreen, fill=ACMGreen},
%   {ACMDarkBlue, fill=ACMDarkBlue}
% }


\begin{document}
\title{Thesis Proposal: Scalable and Practical Tensor Decompositions using Randomization}
\author {
	% Casey Battaglino\thanks{Georgia Institute of Technology Comp. Sci. and Engr. (\email{cbattaglino3@gatech.edu}).}
	Casey Battaglino\thanks{Georgia Institute of Technology Computational Science and Engineering.}
	\\ Advisor: Dr. Richard Vuduc
}
\maketitle
\tableofcontents
% Abstract
% \begin{abstract}
An emerging area of research in computational science considers efficiently computing on data sets that are 
inherently multi-way-- that is, they can be represented by higher-order \emph{tensors}. 
Tensor \emph{decompositions} are a leading method for compressing or finding structure in multi-way 
data sets, and increasing the computational scalability of these decompositions is an open research problem.

It is tempting to think of tensor computations as inherently more costly than matrix computations, e.g. 
with notions such as the `curse of dimensionality.' A key argument of this proposal is that the 
multi-way structure of tensors surprisingly creates numerous avenues for increasing scalability that are not available
in the realm of matrix computations.

Recent developments in randomized numerical linear algebra utilize techniques such as random projection and 
sampling to perform common operations such as low-rank 
decomposition and least squares much faster than traditional methods, with the 
drawback that error bounds must often be restated in probabilistic terms. We propose that these randomized methods can be extended to tensors in a way that uniquely takes advantage of their multi-way structure. 

Towards this goal we develop efficient, high-performance implementations of randomized algorithms for leading tensor decompositions. We demonstrate in our existing work that the CANDECOMP/PARAFAC (CP) decomposition can be performed with randomized least squares, and that the tensor structure actually produces favorable conditions for the algorithm-- in fact, the randomized algorithm can extract more robust features than the leading deterministic algorithm at much lower cost. Our proposal targets scaling up the Tucker decomposition in distributed memory using methods drawn from the randomized SVD, and also targets the Tensor Train decomposition. 
% Finally, we demonstrate how partition structure in power-law graphs can be rapidly discovered using a simple streaming algorithm in distributed memory, and compare scalability and quality to leading alternatives.
\end{abstract}

% \textbf{Thesis Statement:} Randomized methods are uniquely suitable for large-scale tensor and graph computations. This applies both to scalability in shared- and distributed-memory, but also to the robustness and quality of the output. 
% \section{Outline}
% \newpage
\section{Introduction}\label{sec:introduction}
\begin{figure*}
  \centering 
  \includegraphics[width=0.6\linewidth]{thpropfigs/sparseanddense}
  \caption{A visualization of an order 3 ($d=3$) sparse and dense tensor (left and right, respectively.}
  \label{fig:sparseanddense}
\end{figure*}
An emerging area of research in computational science considers efficiently computing on data sets that are 
inherently multi-way-- that is, they can be represented by higher-order \emph{tensors}~\cite{Acar09futuredirections}. 
%
% Whereas matrix and block-matrix methods are widespread and well-understood in computational science, a current challenge is to develop \emph{tensor}-level thinking that deals with the unique challenges of tensor data sets~\cite{Acar09futuredirections}. 
%
% Tensor \emph{decompositions} are a leading method for compressing or finding structure within these 
% data sets, and increasing the computational scalability of these decompositions is an open research problem.
Tensor \emph{decompositions} are a powerful tool for the analysis of multi-way data~\cite{Kolda:2009}, and generally attempt to express an input tensor in a form that is lower-dimensional or of a more desirable structure. The resulting decomposition may be valuable for its interpretability (e.g., for factor analysis), or as a compressed format that alleviates the curse of dimensionality. Two of the most popular decompositions are visualized in~\cref{fig:decomps}, and the storage costs of the decompositions mentioned in this proposal are shown in~\cref{tab:storagecosts}, where the original tensor has dimension $n_k$ in mode $k$ and the decomposition has rank $r_k$ in mode $k$. Typically, $r_k << n_k$.

\begin{table}[h]
\centering\footnotesize
\begin{tabular}{l l}
\textbf{Representation} & \textbf{Storage Cost} \\ \hline
 \text{Original Tensor} &  $\prod_k n_k$ \\
 \text{CP} & $\sum_k r_kn_k$ \\
 \text{Tucker} & $\sum_k r_kn_k + \prod_k r_k$  \\
 \text{Tensor Train} &  $n_1 r_1 + n_d r_d + \sum_{k=2}^{d-1}r_{k-1}r_kn_k$
\end{tabular}
\caption{Storage costs of tensor decompositions in this proposal.}
\label{tab:storagecosts}
\end{table}
%
% It is tempting to think of tensor computations as inherently more costly than matrix computations.
% A key argument of this proposal however is that the multi-way structure of tensors surprisingly creates numerous avenues for increasing scalability that are not available in the realm of matrix computations.

As we can see from this table, a full representation of tensor data can scale arbitrarily large (exponential in its order, if each dimension is of a similar size). It is thus desirable to develop decomposition algorithms that scale in kind. We propose to utilize \emph{randomized} methods towards this end.
Recent developments in randomized numerical linear algebra utilize techniques such as random projection and 
sampling to perform common operations such as low-rank 
decomposition and least squares much faster than traditional methods, with the 
drawback that error bounds must often be restated in probabilistic terms. A central motivation in this proposal is the belief that
the dimensionality of tensors lends itself \emph{particularly} well to these randomized methods. 

Towards this goal we propose to develop efficient, high-performance implementations of randomized algorithms for leading tensor decompositions. We demonstrate in our prior work that the \textsc{Candecomp/Parafac} (CP) decomposition can be performed with randomized least squares, and that the tensor structure actually produces favorable conditions for the algorithm-- in fact, the randomized algorithm can extract more robust features than the leading deterministic algorithm at much lower cost~(\cref{sec:cp}). 

In addition we target scaling up the Tucker decomposition in distributed memory using methods drawn from the randomized SVD~(\cref{sec:tucker}), and propose that the tensor structure also suits both this method and another computation, the Tensor Train decomposition~(\cref{sec:tt}). 
%
\begin{figure}[ht]
  \centering 
  \includegraphics[width=\linewidth]{thpropfigs/decomps}
  \caption{Left: A low-rank matrix decomposition, visualized as a sum of rank-one matrices (top) or as a product of smaller matrices (bottom). Right: The CP decomposition, which is a sum of rank-one tensors (top), and the Tucker decomposition, which is a product of a smaller tensor with $d$ factor matrices in each mode (bottom).}
  \label{fig:decomps}
\end{figure}
%
\begin{figure}[ht]
  \centering 
  \includegraphics[width=0.85\linewidth]{thpropfigs/overdetermine}
  \caption{Intermediate computations in tensor decompositions often involve matrices where one dimension is much larger than the other. This is one example of a feature that is easily exploitable by randomized methods.}
  \label{fig:overdetermine}
\end{figure}
%
\subsection{Proposal}
We propose a line of research to apply sketching techniques to tensor decompositions, and to leverage them in such a way that allows for high-performance algorithms that outperform the state of the art. A key aspect within each decomposition is to show how sketching algorithms can leverage the specific higher-order structure of tensors.
\begin{enumerate}
\item \textsc{Candecomp/Parafac} (CP) Decomposition
\begin{itemize}
	\item The leading method for computing the CP decomposition is CP-ALS, which performs iterative least-squares updates in each mode.
	\item In completed work~\cite{caseyb}, we use methods drawn from \emph{randomized least squares} and apply them to this core computation.
	\item We show that the tensor structure produces conditions favorable for sketching.
	\item We also show that tensor dimensionality produces conditions favorable to sampling, which is not necessarily true for matrices.
	\item We demonstrate scalability on real and synthetic data sets.
	\item We demonstrate the surprising result that this method often works \emph{better} than CP-ALS, returning good solutions more robustly.
	\item Deliverable: this method is published~\cite{caseyb} and released as part of MATLAB Tensor Toolbox~\cite{TTB_Software}\footnote{\url{http://gitlab.com/tensors/tensor_toolbox}}.
\end{itemize}
\item Tucker Decomposition / HOSVD
\begin{itemize}
	\item The leading method for computing the Tucker decomposition is the HOSVD, which performs mode-wise SVDs followed by tensor contractions.
	\item We propose to apply ideas from the Randomized SVD to the HOSVD in distributed memory.
	\item We propose that the tensor structure makes random projection highly efficient because it can be expressed as a series of small tensor contractions without intermediate communication. 
	\item We propose that the tensor structure uniquely allows for high-quality output because the efficiency of projections allows for significant \emph{oversampling} in very little time.
	\item We propose to demonstrate scalability on massive real and synthetic data sets on up to 1000 nodes.
	\item Deliverable: this method will be submitted for publication and included as part of TuckerMPI\footnote{\url{http://tensors.gitlab.io/TuckerMPI/}}.
\end{itemize}
\item Tensor Train Decomposition
\begin{itemize}
	\item The tensor train decomposition is a leading low-rank representation for very high-order tensors.
	\item We propose exploring how the structure of the tensor train computation can be exploited by randomized methods in a way comparable to the previous two methods (CP/Tucker).
	\item We propose demonstration of scalability on massive real and synthetic data sets.
	\item We propose developing a practical framework for the tensor train and quantized tensor train that allows for randomization. 
\end{itemize}
\end{enumerate}

\section{Background and Definitions} \label{sec:background} 
% BACKGROUND sec:background %%%%%%%%%%%%%%%%%%%%%%%%%%%%%%%%%%%%%%%%%%%%%%%%%%%%%%%%%%%%%%%%%%
In this section, we provide information on the necessary tensor properties and operations, and then introduce tensor decompositions and associated randomized methods.
%
\subsection{Tensors}
A tensor is an element in a tensor product of vector spaces. In data analysis, 
it suffices to think about a tensor as a multidimensional array.
We represent a tensor as a Euler script capital letter, e.g., 
$\T{X}\in \R^{n_1 \times \cdots \times n_d}$. 

The number of \emph{modes} (or dimensions) of a tensor is referred to as its \emph{order}, 
denoted by $d$. The values $n_k$ denote the dimensions of a tensor, and we let $n=\sqrt[d]{\prod_k n_k}$, 
so that $n^d = \prod_k n_k$. Thus, the term $n^d$ to more intuitively expresses the number of elements 
 in a tensor as exponential in its order. We also let $n_k^\oslash = n^d / n_k$ represent the product of all 
 dimensions \emph{except} $n_k$.

Let $\mathcal{I} = \{ \V{i} = (i_1,\dots,i_d) \}$ be the set of indices of a tensor. 
We can thus express an individual element of a tensor $\T{X}$ for any multiindex $\V{i}$ as $x_{\V{i}}$. 

The \emph{mode-$k$ fibers} of a tensor are higher-order analogues
of matrix columns and rows.
The mode-$k$ \emph{unfolding} or \emph{matricization} of a tensor 
aligns the mode-$k$ fibers as the columns of an $n_k \times n_k^\oslash$ matrix. 
Assuming 1-indexing, tensor entry $x_{\mathbf{i}}$ then maps to
entry $(i_k, j)$ of $\M{X}_{(k)}$ via the relation: 
\begin{equation}
  \label{eqn:unfoldmapping}
  j = 1+\sum_{\substack{\ell=1\\\ell\neq k}}^d (i_\ell -1)m_\ell, \qquad \text{where} \qquad m_\ell = \prod_{\substack{q=1\\q\neq \ell}}^{\ell-1} n_q.
\end{equation}
The \emph{norm} of a tensor is the square root of the sum of its squared entries, 
e.g., $\|\T{X}\| = \|\M{X}_{(k)}\|_F$ for any $k$. Given a decomposed representation $\T{M}$ of a tensor,
the normalized residual error can be written as:
\begin{equation}
\|\T{X}-\T{M}\|/{\|\T{X}\|}
\end{equation}
Given matrices $\M{A} \in \mathbb{R}^{n_1 \times n_2}$ and $\M{B} \in \mathbb{R}^{m_1
  \times m_2}$, their \emph{Kronecker product} is 
\begin{displaymath}
  \M{A} \otimes \M{B} =
  \begin{bmatrix}
    a_{11} \M{B} & a_{12} \M{B} & \cdots & a_{1n_2} \M{B} \\
    \vdots & \vdots & \ddots & \vdots \\
    a_{n_1 1} \M{B} & a_{n_1 2} \M{B} & \cdots & a_{n_1 n_2} \M{B} \\
  \end{bmatrix}
  \in \R^{n_1 m_1 \times n_2 m_2}.
\end{displaymath}
%
Assuming $n_2 = m_2$, their \emph{Khatri-Rao product}, also known as the
\emph{matching columnwise Kronecker product}, is
\begin{displaymath}
  \M{A} \odot \M{B} = 
  \begin{bmatrix}
    \V{a}_1 \otimes \V{b}_1 & \V{a}_2 \otimes \V{b}_2
    & \cdots &  \V{a}_J \otimes \V{b}_J \in \mathbb{R}^{n_1 m_1 \times n_2}.   
  \end{bmatrix}
\end{displaymath}
Assuming $n_1=m_1$ and $n_2=m_2$, their \emph{Hadamard product} is
$\M{A}\circledast\M{B} \in \mathbb{R}^{n_1 \times n_2}$, the elementwise product of the matrices.
Three useful identities involving the products just defined are:
\begin{align}
(\M{A} \odot \M{B})^\trans(\M{A} \odot \M{B}) &= \M{A}^\trans\M{A} \circledast \M{B}^\trans\M{B}, \label{eq:KRGram} \\ 
\M{A}\M{B} \otimes \M{C}\M{D} &= (\M{A} \otimes \M{C})(\M{B} \otimes \M{D}) \label{eqn:krondist},\quad\text{and} \\
\M{A}\M{B} \odot \M{C}\M{D} &= (\M{A} \otimes \M{C})(\M{B} \odot \M{D}). \label{eqn:krdist}
\end{align}

%
The \emph{mode-$k$ tensor-times-matrix product} (TTM) is a contraction
between a matrix and a tensor in its $k$th mode.
\begin{equation}
\label{eqn:ttm}
\T{Y} = \T{X} \times_k \M{A} \quad \Leftrightarrow \quad \M{Y}_{(k)} = \M{A}\M{X}_{(k)}.
\end{equation}
We can also write this element-wise as
\begin{equation}
  \label{eqn:contractionelem}
  y_{i_1i_2\cdots i_{k-1}j i_{k+1}\cdots i_d} = \sum_{i_k}x_{i_1\cdots i_d}u_{j i_k}
\end{equation}
We will use bracket notation to denote multiple products, e.g. 
$\T{X} \times \{\Mn{U}{k}\}$ refers to $\T{X}$ multiplied by $\Mn{U}{k}$ 
for every $k=1,\dots,d$. The result is invariant to which order the TTMs are performed in, 
providing the modes are unique. If a tensor can be written as a series of 
mode-$k$ products, its mode-$k$ matricization has a particular structure~\cite{Kolda:2009}:
\begin{align}
  \begin{split}
    \label{eqn:ttensor}\T{Y} & ={}  \T{X} \times \{\Mn{U}{k}\} \quad \Leftrightarrow \\
    \M{Y}_{(k)} & ={}  \Mn{U}{k} \M{X}_{(k)} 
    (\Mn{U}{d} \otimes \cdots \otimes \Mn{U}{k+1} \otimes \Mn{U}{k-1} \otimes \cdots \otimes \Mn{U}{1})^\trans.
  \end{split}
\end{align}
%
\begin{table}[ht]
  \centering%\footnotesize
  \label{tab:notation}
  \begin{tabular}{cl}
    \toprule
    Notation & Definition \\
    \midrule
    $\M{X}_{(k)}$ & mode-$k$ unfolding of $\T{X}$ \\
    $\T{X}\times_k \M{U}_k$ & Tensor-Times Matrix Multiplication (TTM) \\
    $\M{A}\otimes\M{B}$ & Kronecker Product \\
    $\M{A}\odot\M{B}$ & Khatri-Rao Product \\
    $\M{A}\circledast \M{B}$ & Hadamard Product \\
    \midrule
    $d$ & Number of modes (order) of a tensor \\
    $n_k$ & Size of mode $k$ of tensor $\T{X}$ \\
    $r_k$ & Rank (core size) of mode $k$\footnote{For the CP decomposition there is only one rank $r$} \\
    $p_k$ & Number of processors along mode $k$ \\
    $s_k$ & Sketch size of mode $k$ \\
    $n,r,p,s$ & $\sqrt[d]{\prod n_k}, \sqrt[d]{\prod r_k}, \sqrt[p]{\prod p_k}, \sqrt[d]{\prod s_k}$ \\
    $n_k^\oslash,r_k^\oslash,p_k^\oslash,s_k^\oslash$ & $n^d/n_k,r^d/r_k,p^d/p_k,s^d/s_k$ \\
    % \midrule
    % $\Mn{U}{k}\in\mathbb{R}^{n_k\times r_k}$ & Tucker factor matrix for mode $k$ \\
    % $\M{\Omega}_k\in\mathbb{R}^{n_k \times s_k}$ & Random sketch matrix for mode $k$ \\
    % $\T{G}\in\mathbb{R}^{r_1\times\cdots\times r_d}$ & Tucker core \\
    % $\T{M} = \{\T{G},\{\M{U}_k\} \}$ & Tucker approximation of $\T{X}$ \\
    % $\bar{\T{X}}$ & Local subtensor of $\T{X}$ \\
    \bottomrule
  \end{tabular}
  \caption{Basic tensor notation used in this proposal.}
\end{table}
%
\subsection{CP Decomposition}
The CP tensor decomposition aims to approximate an order-$d$ tensor as
a sum of $r$ rank-one
tensors~\cite{hitchcock-sum-1927, CANDECOMP, PARAFAC, Kolda:2009}:  
\begin{equation}
\label{eqn:cpform}
\T{X} \approx \T{\tilde{X}} = \sum_{k=1}^d \V{a}_k^{(1)} \circ \V{a}_k^{(2)} \circ \cdots \circ \V{a}_k^{(d)},
\end{equation}
where \emph{factor vector} $\V{a}_r^{(k)}$ has length $n_k$. 
Each rank-one tensor is called a \emph{component}.
The collection of all factor vectors for a given mode is called a 
\emph{factor matrix}:
\begin{displaymath}
  \Mn{A}{k} =
  \begin{bmatrix}
    \MnC{A}{k}{1} &
    \MnC{A}{k}{2} &
    \cdots &
    \MnC{A}{k}{r}
  \end{bmatrix}
  \in\R^{n_k \times r}.
\end{displaymath}
The mode-$k$ matricization of $\T{\tilde{X}}$ can be written in terms
the factor matrices as
\begin{equation}\label{eq:Zn}
  \M{\tilde{X}}_{(k)} = \M{A}^{(k)}\M{Z}^{(k)\trans}
  \qtext{where}
  \M{Z}^{(k)}=\M{A}^{(d)}\odot \cdots  \M{A}^{(k+1)}\odot
  \M{A}^{(k-1)} \odot \cdots \odot \M{A}^{(1)}.   
\end{equation}
We may alternatively represent \cref{eqn:cpform} by normalizing all
the factor vectors to unit length and expressing the product of the
normalization factors as a scalar weight $\lambda_r$ for each
component: 
\begin{equation}\label{eq:cpformlambda}
  \T{\tilde{X}} = \sum_{k=1}^d \lambda_k \; \V{a}_k^{(1)} \circ
  \V{a}_k^{(2)} \circ \cdots \circ \V{a}_k^{(d)}.
\end{equation}

\subsubsection{CP-ALS}

The standard method for fitting the CP model is alternating least
squares (CP-ALS) \cite{PARAFAC,Kolda:2009}. The method alternates
among the modes, fixing every factor matrix but $\Mn{A}{k}$ and
solving for it. From \cref{eq:Zn}, 
we see that we can find $\Mn{A}{k}$ by solving the linear least squares
problem given by
\begin{equation}
\label{eq:lls}
\argmin_{\M{A}^{(k)}} \|\M{X}_{(k)} - \M{A}^{(k)}\M{Z}^{(k)\trans}\|_F.
\end{equation}
In CP-ALS, we work with the normal equations for \cref{eq:lls}:
\begin{displaymath}
\M{X}_{(k)}\M{Z}^{(k)} = \M{A}^{(k)}(\M{Z}^{(k)\trans}\M{Z}^{(k)}),
\end{displaymath}
and solve for $\M{A}^{(k)}$ for given $\M{X}_{(k)}$ and $\M{Z}^{(k)}$.
By identity \cref{eq:KRGram}, we have 
\begin{displaymath}
\M{Z}^{(k)\trans}\M{Z}^{(k)} = \M{A}^{(d)\trans}\M{A}^{(d)} \circledast \dots \circledast \M{A}^{(k+1)\trans}\M{A}^{(k+1)} \circledast \M{A}^{(k-1)\trans}\M{A}^{(k-1)} \circledast \cdots \circledast \M{A}^{(1)\trans}\M{A}^{(1)}.
\end{displaymath}

The CP-ALS algorithm~\cite{Kolda:2009} is presented
in~\cref{alg:cpals}. Note the step where vector $\bm{\lambda}$ stores
normalization values of each column so that the final approximation is
as in \cref{eq:cpformlambda}; this normalization helps alleviate
issues due to scaling ambiguity.

The initialization of the factor matrices
can impact the performance of the algorithm.
There are many possible ways to do the the initialization.
One way is to
initialize is to set $\Mn{A}{k}$ to be the leading $r$ left singular
vectors of the mode-$k$ unfolding, $\M{X}_{(k)}$, and we call this
HOSVD initialization, as it corresponds to the factor matrices
in the rank-$(r{\times} {\cdots} {\times} r)$ HOSVD (see~\cref{sec:hosvd}). 
A less expensive but less effective initialization is to
choose random factor matrices.

\begin{algorithm}
  \caption{CP-ALS}
  \label{alg:cpals}
  \begin{algorithmic}[1]\footnotesize
    \Function{$[\bm{\lambda},\set{\M{A}^{(n)}}]=$ CP-ALS}{$\T{X},R$}\Comment{$\T{X}\in\mathbb{R}^{I_1\times \cdots \times I_N}$}
    \State \label{line:cpals:init} Initialize factor matrices $\M{A}^{(2)}, \dots, \M{A}^{(N)}$
    \Repeat
    \For{$n=1,\dots, N$}
      \State $\M{V} \gets \M{A}^{(N)\trans}\M{A}^{(N)} \circledast \dots \circledast \M{A}^{(n+1)\trans}\M{A}^{(n+1)} \circledast \M{A}^{(n-1)\trans}\M{A}^{(n-1)} \circledast \cdots \circledast \M{A}^{(1)\trans}\M{A}^{(1)}$\label{line:cpals:Gram}
      \State \label{line:cpals:KR} $\M{Z}^{(n)} \gets \M{A}^{(N)}\odot \cdots  \odot \M{A}^{(n+1)}\odot \M{A}^{(n-1)} \odot \cdots \odot \M{A}^{(1)}$
      \State \label{line:cpals:MTTKRP} $\M{W} \gets \M{X}_{(n)}\M{Z}^{(n)}$
      \State \label{line:cpals:solve} Solve $\M{A}^{(n)}\M{V} = \M{W}$ for $\M{A}^{(n)}$        
      \State Normalize columns of $\M{A}^{(n)}$ and update $\bm{\lambda}$
    \EndFor
    \Until termination criteria met
    \State \textbf{return} $\bm{\lambda}$, factor matrices $\set{\M{A}^{(n)}}$
    \EndFunction
  \end{algorithmic}
\end{algorithm}


\subsection{The Tucker Decomposition} \label{sec:hosvd} 
\begin{figure}[htbp]
  \centering
\begin{tikzpicture}[scale=0.5,namenode/.style={scale=.75}]
	\def\ix{3} %
	\def\iy{3} %
	\def\iz{2.5} %
	\def\corescale{1.75}
	\def\rx{\ix/\corescale}
	\def\ry{\iy/\corescale}
	\def\rz{\iz/\corescale}
	\coordinate (XFrontLowerLeft) at (0,0);
	\draw (XFrontLowerLeft) rectangle ++ (\ix,\iy); %
	\begin{scope}[shift={(XFrontLowerLeft)},canvas is zx plane at y=\iy,rotate=90]
	  \draw (0,0) rectangle ++ (\ix,\iz); %
	\end{scope}
	\begin{scope}[shift={(XFrontLowerLeft)},canvas is zy plane at x=\ix,rotate=90]
	  \draw (0,0) rectangle ++ (\iy,\iz); %
	\end{scope}
	\node[namenode] at ($(XFrontLowerLeft) + (0.5*\ix, 0.5*\iy)$)  {$\T{X}$};
	\coordinate (ApproxCtr) at ($(XFrontLowerLeft) + (\ix+0.4*\iz,0.75*\iy) + (0.75,0)$);

	\node[namenode] at (ApproxCtr) {$\approx$};
	\coordinate (U1LowerLeft) at ($(ApproxCtr) - (0,0.75*\iy) + (0.75,0)$);
	\draw (U1LowerLeft) rectangle ++ (\ry,\iy);

	\node[namenode] at ($(U1LowerLeft)+(0.5*\ry, 0.5*\iy)$)  {$\Mn{U}{1}$};
	\coordinate (GFrontLowerLeft) at ($(U1LowerLeft) + (\ry+0.5,1)$);
	\draw (GFrontLowerLeft) rectangle ++ (\rx,\ry);
	\begin{scope}[shift={(GFrontLowerLeft)},canvas is zx plane at y=\ry,rotate=90]
	  \draw (0,0) rectangle ++ (\rx,\rz);
	\end{scope}
	\begin{scope}[shift={(GFrontLowerLeft)},canvas is zy plane at x=\rx,rotate=90]
	  \draw (0,0) rectangle ++ (\ry,\rz);
	\end{scope}
	\node[namenode] at ($(GFrontLowerLeft)+(0.5*\rx,.5*\ry)$)  {$\T{G}$};
	\coordinate (U2LowerLeft) at ($(GFrontLowerLeft) + (\rx+\rz*0.4+0.5,0.5)$);
	\draw (U2LowerLeft) rectangle ++ (\ix,\rx); %

	\node[namenode] at ($(U2LowerLeft)+(0.5*\ix,0.5*\rx)$)  {$\Mn{U}{2}$};
	\coordinate (U3LowerLeft) at ($(GFrontLowerLeft) + (0.5,\ry+.8)$);
	\begin{scope}[shift={(U3LowerLeft)},canvas is zx plane at y=0,rotate=90]
	  \draw (0,0) rectangle ++ (\rz,\iz); %
	\end{scope}
	\node[namenode] at ($(U3LowerLeft)+(1.2,0.5)$) {$\Mn{U}{3}$};
\end{tikzpicture}
  \caption{Tucker decomposition of 3rd-order tensor ($d=3$).}
  \label{fig:tucker}
\end{figure}
%%% Local Variables:
%%% mode: latex
%%% TeX-master: t
%%% End:
 %%%%%% Tucker Figure %%%%%%
The Tucker decomposition~\cite{Tu66} approximates a tensor with a core tensor contracted with 
matrices in each mode:
\begin{displaymath}
  \T{X} \approx \T{M} = \T{G} \times_1 \Mn{U}{1} \times_2 \Mn{U}{2} \cdots \times_d \Mn{U}{d} = \T{G} \times \{\Mn{U}{k}\},
\end{displaymath}
where
$\T{G}$ is a dense core of size $r_1 \times r_2 \times \cdots \times r_d$, and the factor matrices 
$\Mn{U}{k}$ have size $n_k \times r_k$ for $k=1, \dots, d$.

The \hosvd~\cite{Lathauwer00amultilinear} is a method for computing the Tucker decomposition that computes a series 
of SVDs of
unfolded tensors to compute the orthogonal factor matrices $\Mn{U}{k}$ whose 
columns approximately span the columns of the unfolded tensor $\M{X}_{(k)}$ for each mode $k$. We will refer to this computation as the Mode-wise Truncated Fiber Space Basis Computation (\MTFSBC).
Following this operation, the resulting factor matrices are applied to the input tensor to compress the original data into the core $\T{G}$: $\T{G} \gets \T{X} \times \{ \Mn{U}{k} \}$. 

Just as we would compress a matrix by truncating its SVD, we can truncate the factors 
$\Mn{U}{k}$ so that $\T{G}$ is a smaller core ($r_k < n_k$), a method we refer to as the 
\emph{truncated} \hosvd (\thosvd), presented in~\cref{alg:sthosvd}.
This is particularly effective for compression; as with $n$, we let $r=\sqrt[d]{\prod r_k}$, and so 
the core is exponentially smaller than $\T{X}$, a factor of $(\frac{n}{r})^d$. 
The total compression ratio includes the factor matrices:
\begin{equation}
\label{eqn:compression}
n^d / \left(r^d + \sum_{k=1}^d n_k r_k\right).
\end{equation}

The HOSVD can be
implemented by forming the $n_k \times n_k$ Gram matrix $\M{S}_k = \M{X}_{(k)}\M{X}_{(k)}^\trans$ 
and computing its eigendecomposition. The eigenvectors of $\M{S}_k$ correspond to the left
singular vectors of $\M{X}_{(k)}$, and $\lambda_j(\M{S}_k) = \sigma_j(\M{X}_{(k)})^2$. 
This Gram computation is utilized by TuckerMPI, which performs a distributed Gram matrix 
computation followed by a local eigendecomposition~\cite{AuBaKo16}.
We use the Gram variant because we 
assume that dimensions are reasonably-sized, e.g., $n_k \leq 10^4$ for all $k$.

\begin{algorithm}[htb]
  \caption{\thosvd\Comment{Gram Variant}}
  \begin{algorithmic}[1]
    \Procedure{\thosvd}{$\T{X}$, $(r_1,\dots,r_d)$}
    \For{$k=1,\dots,d$}
    % \State $[\M{U},\M{\Sigma}] \gets \text{SVD}(\M{X}_{(k)})$
    \State $\M{S}_k \gets \M{X}_{(k)}\M{X}_{(k)}^\trans $ \Comment{Gram}
    \State $[\M{U},\M{\lambda}] \gets \text{eig}(\M{S}_k)$ \Comment{Eigensolve}
    % \State $r_k \gets $ min $j$ s.t. $\sum_{r>j} \lambda_r \leq \epsilon^2\|\T{X}\|^2/d$
    \State \label{line:sthosvd:svd} $\Mn{U}{k} \gets \text{ leading $r_k$ eigenvectors in $\M{U}$}$
    \EndFor
    \State \label{line:sthosvd:trunc} $\T{G} \gets \{ \T{X} \times \Mn{U}{k}^\trans\}$ \Comment{TTM (Core Formation)}
    \State $\T{M} \gets \{ \T{G}, \set{\Mn{U}{k}} \}$ \label{line:sthosvd:ttm}
    \State \Return $\T{M}$
    \EndProcedure
  \end{algorithmic}
  \label{alg:sthosvd}
\end{algorithm}
%
A new variant of the \thosvd is the \emph{sequentially} truncated \hosvd (\sthosvd)~\cite{sthosvd}, 
which is implemented in TuckerMPI~\cite{AuBaKo16}. Whereas the \thosvd forms the core 
after performing the \MTFSBC for all modes (in~\cref{line:sthosvd:ttm}), 
the \sthosvd performs a \MTFSBC for a single mode followed by a core-compression step in that mode,
such that the $k$th TTM occurs within the $k$th iteration of the for loop. 
The working size of the tensor thus reduces by a 
factor of $n_k / r_k$ at each iteration, with an equivalent reduction in the cost of subsequent SVD steps.
%
\subsection{Tensor Train Decomposition}
The Tensor Train decomposition (TT) factors an order-$d$ tensor into a set of $d$ tensors. The $k$th tensor is denoted $G^{(k)}$, where $G^{(1)}$ and $G^{(d)}$ are of order 2, and all other tensors are of order 3. 

The input tensor can then be reconstructed via the relation:
\begin{equation}
\T{X} \approx \M{G}^{(1)} \times_{2}^1 \T{G}^{(2)} \times_{3}^1 \T{G}^{(3)} \times_3^1 \dots
\times_3^1 \T{G}^{(d-1)}\times_3^1 \M{G}^{(d)} 
\end{equation}
where $\T{X} \times_a^b \T{Y}$ can be computed as $\M{X}_b \M{Y}_a$, or more generally as 
\begin{align*}
\T{X}(i_1,\dots,i_d) & \approx  
\sum_{{k_1}=1}^{r_1} \sum_{{k_2}=1}^{r_2}\cdots\sum_{{k_{d-1}}=1}^{r_{d-1}} \T{G}^{(1)}(i_1,k_1)\T{G}^{(2)}(k_1,i_2,k_2) & \\ & \cdots\T{G}^{(d-1)}(k_{d-2},i_{d-1},k_{d-1})\T{G}^{(d)}(k_{d-1},i_d)
\end{align*}
where $\M{G}^{(1)}\in \R^{r_1\times n_1}$, $\T{G}^{(k)} \in \R^{r_{k-1}\times n_k \times r_k}$ for $1< k< d$ and $\M{G}^{(d)}\in \R^{R_{d-1}\times n_d}$. The relationship between tensors can be better understood in terms of a tensor network diagram. In this visualization, each node represents a tensor, and each line represents a mode, with the associated mode size. Lines that connect tensors represent the common mode that can be contracted to reconstruct the input tensor:
\begin{figure}[htbp]
    \center\includegraphics[width=0.80\linewidth]{thpropfigs/tensortrain}
    \caption{Tensor Train Decomposition}
\end{figure}
Computing this representation involves a straightforward recursive application of matricization and the skinny SVD.
\subsection{Randomized Least Squares}
\subsection{Randomized SVD}
%
%
%
%
%
%
%
\section{Literature Survey}
A broad survey of tensor decompositions is provided by Kolda and Bader~\cite{Kolda:2009}. Additional surveys outline specific tensor techniques in machine learning~\cite{nikossurvey}, quantum chemistry~\cite{quantumsurvey}, latent variable models~\cite{Anandk}, and neuroscience~\cite{eegsurvey}. In this proposal we will focus on scalable implementations of three of the most popular tensor decompositions: the \textsc{Candecomp/Parafac} (CP) decomposition~\cite{hitchcock-sum-1927, CANDECOMP, PARAFAC}, the Tucker decomposition~\cite{Tu66}, and the Tensor Train decomposition~\cite{tensortrain}. 

\paragraph{Sketching}
Sketching is a technique for solving linear algebra problems by constructing a smaller problem whose solution is a reasonable approximation to the original problem with high probability~\cite{sketching}. For instance, a large matrix may be sketched by applying random sampling or random projections to form a smaller sketch matrix. While this approach is suitable for a wide variety of problems, we will focus on two essential building blocks: randomized least squares~\cite{rokhlintygert,DrMaMuSa11,blendenpik}, and the randomized SVD~\cite{halko}. 




% \section{Literature Review}

\section{Randomized CP Decomposition} \label{sec:cp} 
The CANDECOMP/PARAFAC (CP) tensor decomposition is an important tool
for data analysis in applications such as chemometrics~\cite{MuStGrBr13}, biogeochemistry~\cite{JaCaYa14},
neuroscience~\cite{AcBiBiBr07,DaGiCaWa13,CoLiKuGo15}, cyber traffic analysis~\cite{MaGuFa11}, and many others.
%
We consider the problem of accelerating the alternating least squares (CP-ALS) algorithm using randomization.

Because randomized methods have been used successfully for solving
linear least squares problems~\cite{DrMaMuSa11,blendenpik,sketching}, it is natural that they might prove
beneficial to CP-ALS since its key kernel is the solution of a least
squares problem. However, the CP-ALS least squares subproblem has a
special structure that already greatly reduces its cost (see~\cref{eqn:nicegram}), so it is not obvious that sketching would be beneficial.
Nevertheless, we find that our randomized algorithms significantly reduce the memory and computational overhead of the CP-ALS process for dense tensors to increase performance~(\cref{fig:tvsfmain}) and moreover positively impact algorithmic robustness~(\cref{fig:tammyexperiment}).
To the best of our knowledge, this is the first successful application of matrix sketching methods in the context of CP.
Our contributions are as follows:
\begin{itemize}
\item 
The least squares coefficient matrix in the CP-ALS subproblem is a Khatri-Rao product of factor matrices. 
Our randomized algorithm prefers \emph{incoherent} matrices.
We prove that the coherence of the Khatri-Rao product is bounded 
above by the product of the coherence of its factors:
\begin{lemma} \label{lem:krp}
Given $\M{A} \in \mathbb{R}^{I \times J}$ and $\M{B} \in \mathbb{R}^{K \times L}$, $\mu(\M{A} \odot \M{B}) \leq \mu(\M{A})\mu(\M{B})$.
\end{lemma}
\item We introduce the CPRAND algorithm that uses a randomized least squares solver for the subproblems in CP-ALS and never explicitly forms the full Khatri-Rao matrices used in the subproblems.
We also introduce the complementary CPRAND-MIX algorithm that employs efficient \emph{mixing} to promote incoherence and thereby improves the robustness of the method.
\item We derive a novel, lightweight stopping condition that estimates
  the model fit error, and we prove its accuracy using Chernoff-Hoeffding bounds.
\item We demonstrate the speed and robustness of our algorithms over a
  large number of synthetic tensors as well as real-world data
  sets. In comparison with CP-ALS, CPRAND is faster and much less sensitive to
  the starting point. (For instance, see~\cref{fig:tvsfmain} for speed and~\cref{fig:tammyexperiment} for robustness).
\end{itemize}
\begin{figure}[h]
  \centering 
  \includegraphics[width=0.7\linewidth]{figs/tammyexperiment}
  \caption{To demonstrate robustness, we show fit and score over 30 runs of a standard CP-ALS algorithm vs. our algorithm (CP-ALS-RAND with mixing) on a $5 \times 201 \times 61$ tensor drawn from chemometrics. Fit corresponds to the normalized residual error~(\cref{eqn:residual}) subtracted from 1. Score measures how well the recovered factors correspond to the best known actual factors.}
  \label{fig:tammyexperiment}
\end{figure}
We give an example of our methods' fast time to solution in \cref{fig:tvsfmain},
comparing CPRAND and CPRAND-MIX with CP-ALS.
For the CPRAND methods, we use 100 sampled rows for each least squares solve.
The randomized methods converge much more quickly, in only a few
iterations. 
The fit is not monotonically increasing for the randomized methods due
to (small) variations in the solution to each randomized subproblem.
\begin{figure}[tbhp]
  \centering \subfloat[Random $300\times300\times300$ tensor]{
    \begin{tikzpicture}
      \begin{axis}[width=.49\textwidth, height=1.8in, grid=major,
        xlabel={time ($s$)}, ylabel={fit},
        xmin=0,xmax=20,ymin=0.9,ymax=1,legend style={font=\smaller},
        legend pos=south east,style={thick}
        ]
        \addplot[color=cpalsncolor,mark=*,mark size=1pt] table [x=cpt,y=cpf] {./data/tvsf300.dat};
        \addplot[color=cprandcolor,mark=*,mark size=1pt] table [x=cprandt,y=cprandf] {./data/tvsf300.dat};
        \addplot[color=cprandfcolor,mark=*,mark size=1pt] table [x=cprandfftt,y=cprandfftf] {./data/tvsf300.dat};
        \addplot[mark=none, black, samples=2,very thin,dashed] coordinates {(0,0.99) (20,0.99)};
      \end{axis}
    \end{tikzpicture}
    \label{fig:tvsf}}
  \subfloat[Random $80\times80\times80\times80$ tensor]{
    \begin{tikzpicture}
      \begin{axis}[width=.49\textwidth, height=1.8in, grid=major,
        xlabel={time ($s$)},
        xmin=0,xmax=20,ymin=0.9,ymax=1,legend style={font=\smaller},
        legend entries={CP-ALS,CPRAND,CPRAND-MIX}, legend pos=south east,style={thick}
        ]
        \addplot[color=cpalsncolor,mark size=1pt,mark=*] table [x=cpt,y=cpf] {./data/tvsf80.dat};
        \addplot[color=cprandcolor,mark size=1pt,mark=*] table [x=cprandt,y=cprandf] {./data/tvsf80.dat};
        \addplot[color=cprandfcolor,mark size=1pt,mark=*] table [x=cprandfftt,y=cprandfftf] {./data/tvsf80.dat};
        \addplot[mark=none, black, samples=2,very thin,dashed] coordinates {(0,0.99) (20,0.99)};
      \end{axis}
    \end{tikzpicture}
    \label{fig:tvsf2}}
  \caption{Runtime comparison for fitting the CP tensor decomposition
    on random synthetic tensors generated
    to have rank 5, factor collinearity of 0.9, and 1\% noise.
    We compare a single run of three methods using a target rank of 5.
    CPRAND and CPRAND-MIX use random initialization,
    100 sampled rows for each least squares solve. 
    CP-ALS uses HOSVD initialization.
    The marks indicate each iteration.
    The
    thin dashed black line represents a fit of 99\%, which is the best
    we expect when the noise is 1\%.} 
    \label{fig:tvsfmain}
  \end{figure}


\section{Randomized Tucker Decomposition} \label{sec:tucker} 
A standard method for computing the Tucker decomposition is the Higher-Order SVD 
(\hosvd)~\cite{Lathauwer00amultilinear}, a generalization of the singular value 
decomposition to tensors.
% 
As mentioned in~\cref{sec:hosvd}, a direct approach to the \hosvd involves computing a series of SVDs of 
unfolded tensors to compute the orthogonal factor matrices whose 
columns approximately span the columns of the unfolded tensor (i.e., the 
mode-$k$ fiber space). We will refer to this computation as the Mode-wise Truncated Fiber Space Basis Computation (\MTFSBC).
Following this operation, the resulting factor matrices are applied to the input tensor to compress the original data.
%

%
We consider a distributed-memory application for large dense tensors, as might arise in large-scale 
scientific simulations. For instance, we consider a five-way tensor representing three spatial dimensions, 
a time dimension, and a number of different variables. A user may wish to examine this data on a local 
machine with limited memory or to store many simulations in limited storage, and the \hosvd has been 
shown to be a highly scalable method for performing this compression.
The current state of the art for distributed \hosvd is TuckerMPI\footnote{\url{http://tensors.gitlab.io/TuckerMPI/}}, 
based on the work of Austin, Ballard, and Kolda~\cite{AuBaKo16}. As in the Gram approach of~\cref{sec:hosvd}, the \MTFSBC kernel is performed with a distributed Gram matrix computation followed by a local eigensolve. The factors are then applied to the input tensor using a distributed tensor-matrix multiplication, forming the dense core tensor. The core tensor and factor matrices can be used to construct an approximation to the original data or any subset thereof.

\paragraph{Distributed implementation}
% \TODO{Suggest $\ell$ rather than $q$ as processor index.}
We will assume the tensor is distributed according to a \emph{Cartesian} distribution.
Given a processor grid $p_1 \times \cdots \times p_d$, we define notation such that the total number of processors
is $p^d = \prod_{k=1}^d p_k$. We use an overbar to denote a local object,
so each process owns a tensor $\bar{\T{X}}$ which is a subtensor of $\T{X}$.
Assuming for 
notational convenience that $n_k$ evenly divides $p_k$, the size of each subtensor 
is $\frac{n_1}{p_1} \times \cdots \times \frac{n_d}{p_d}$, so process $(\ell_1,\dots,\ell_d)$ 
owns the subtensor  with range $\mathbf{i} = ((\ell_1-1)\frac{n_1}{p_1}:\ell_1\frac{n_1}{p_1}-1,\dots)$. 
Furthermore, if we matricize a subtensor in mode $k$ to produce $\bar{\M{X}}_{(k)}$, 
the result is a submatrix in the full matricization $\M{X}_{(k)}$. 
This is visualized in~\cref{fig:tensor_block_dist}, though the columns owned by a processor are not always contiguous as shown here.
Collective communications in this 
model are then implemented either on block-rows or block-columns. 

The column-communicator for process $(\ell_1,\dots,\ell_d)$ in mode $k$ is the set of $p_k$ 
processes $(\ell_1,\dots,\ell_{k-1},*,\ell_{k+1},\dots,\ell_d)$, whereas the row-communicator is 
the set of $p_k^\oslash$ processes $(*,\dots,*,\ell_k,*,\dots,*)$. 
These communicators are visualized in~\cref{fig:tensor_comm}.
% \begin{figure}[]
  \centering
  \begin{tikzpicture}[scale=.3,textnode/.style={scale=.6}]
    \def\ix{3} %
    \def\iy{4} %
    \def\iz{2} %
    \def\sx{1.5} %
    \def\sy{1}
    \def\sz{2}
    \def\offset{0.25}
    \def\myquad{0.5}
    \draw (0,0) grid[xscale=\sx,yscale=\sy] (\ix,\iy); %
    \node[textnode,below] at (0.5*\ix*\sx,0) {$\leftarrow \; n_2 \; \rightarrow$};
    \node[textnode,rotate=90,above] at (0, 0.5*\iy*\sy)  {$\leftarrow \; n_1 \; \rightarrow$};
    \begin{scope}[shift={(XFrontLowerLeft)},canvas is zx plane at y=\iy*\sy,rotate=90]
      \draw (0,0) grid[xscale=\sx,yscale=\sz] (\ix,\iz); %
    \end{scope}
    \begin{scope}[shift={(XFrontLowerLeft)},canvas is zy plane at x=\ix*\sx,rotate=90]
      \draw (0,0) grid[xscale=\sy,yscale=\sz] (\iy,\iz); %
      \node[textnode,rotate=45,below] at (0,0.5*\iz*\sz) {$\leftarrow \; n_3 \; \rightarrow$};
    \end{scope}
  \end{tikzpicture}
  \caption{Tensor  on $4 {\times} 3 {\times} 2$ processor grid.}
  \label{fig:tensor_block_dist}
\end{figure}
\begin{figure}
  \centering 
  \includegraphics[width=0.6\linewidth]{figs/cartesian}
  \caption{A tensor on a $4\times3\times2$ processor grid. The matricized subtensor shown in blue maps to a submatrix in the distributed matricized tensor, as illustrated on the right. 
  %\GB{I'm changing ``block'' to ``submatrix'' because it's not necessarily a contiguous block if you unfold in a middle dimension (the rows are contiguous but the columns may not be).}
  }
  \label{fig:tensor_block_dist}
\end{figure}
\begin{figure}
  \centering 
  \includegraphics[width=0.55\linewidth]{figs/communicators}
  \caption{A process-centric view of~\cref{fig:tensor_block_dist}. The column-communicator in mode $k$ is the set of $p_k$ processes that own blocks in the same column of $\M{X}_{(k)}$, while the row-communicator is the set of $p_k^\oslash$ processes that own blocks in the same row.}
  \label{fig:tensor_comm}
\end{figure}

\paragraph{Proposal}
We propose efficiently performing the \MTFSBC kernel in distributed memory by applying methods 
from randomized numerical linear algebra. We propose that a series of local multiplications followed by a small 
global communication can result in a \emph{sketch} of the original tensor that fits in local memory. 
% with comparable error. 
% Though randomized methods work with only a subset of the data, the errors are very close.
Finally, we propose to apply this to the efficient computation of a Tucker decomposition for the compression of scientific data.
%The following is a more detailed breakdown of our contributions:
%Using this sketch we can, with the help of our theoretical results, compute an \hosvd that still satisfies (with high probability) the error tolerance. 
% \TODO{Include a plot of compression ratio / speed between TuckerMPI and sketching method.}
We summarize our proposed contributions as follows:
\begin{itemize}
  \item Implement and develop a cost analysis for an efficient sketch-based \MTFSBC in 
  distributed memory, using randomization to reduce computation and communication. This method would exploit the Kronecker structure of the sketching operation to reduce computation.
  \item Introduce a new parallel method for a sequence of tensor-times-matrix operations that trades extra computation for reduced communication and show how and when this can be used to further increase the efficiency of sketching. 
  \item Apply our kernel to the Tucker tensor decomposition, benchmark our method on large-scale scientific 
  data sets, and demonstrate scaling on synthetic data sets to improve upon the fastest known 
  traditional implementation.
  \item Provide theory to show how the error introduced through randomization can be compensated for.
  % No previous work has considered this issue, but it is key to preserving data accuracy.
\end{itemize}

\paragraph{Sketching as a series of TTMs}
Consider the algorithm for estimating the range of a matrix in~\cref{alg:rrf}. 
Similarly, to estimate the range of an unfolded tensor we can compute
$\M{Y}_{(k)} \gets \M{X}_{(k)} \M{\Omega}$, where $\M{\Omega}$ 
is a $n_k^\oslash \times \ell$ random matrix. This approach has previously been taken for tensors~\cite{bigtens,DBLP:conf/acssc/VervlietDL16,Navasca2015}. The cost of this step would be 
comparable to a TTM, which could easily become a bottleneck.
However, suppose instead that we multiply with a \emph{Kronecker} product of $d-1$ small random matrices of size $s_k \times n_k$:
\begin{equation}
  \label{eqn:sketchasgemm}
  \M{Y}_{(k)} \gets \M{X}_{(k)} \Big( \M{\Omega}_1 \otimes \cdots \otimes \M{\Omega}_{k-1} \otimes \M{\Omega}_{k+1} \otimes \dots \otimes \M{\Omega}_d \Big)^\trans
\end{equation}
Using~\cref{eqn:ttensor}, we can see that this is equivalent
to performing a series of TTMs with these small matrices:
\begin{equation}
  \label{eqn:sketchasttm}
  \T{Y} \gets \T{X} \times \Big\{ \M{\Omega}_j^\trans \Big\}_{j \neq k}.
\end{equation}

We propose to use this method because it uses far fewer flops than previous work, and the operation can be efficiently performed in distributed memory as a series of distributed TTMs. 

\paragraph{Reducing communication of a TTM sequence}
We present a novel method for computing a TTM sequence in distributed memory which outperforms standard methods when the input matrices have a large aspect ratio (so that the output tensor is much smaller than the input tensor). 
This method significantly reduces communication in certain cases but may be more expensive in others.
It also performs more computation than the standard approach, so we use a performance model to guide its usage.

Consider computing a sequence of TTMs $\T{Y} \gets \T{X} \times_1 \M{A} \times_2 \M{B} \times_3 \M{C}$. Previous implementations of the TTM~\cite{AuBaKo16} first perform the local multiply $\bar{\M{A}}\bar{\M{X}}_{(1)}$ on each process, forming an intermediate result of size $s_1 \times n_1^\oslash / p_1^\oslash$. This is followed by a Reduce-Scatter among the column communicator so that each process owns a local result of size $s_1/p_1 \times n_1^\oslash / p_1^\oslash$, and the process is repeated for the following modes $k=2$ and $k=3$. 
If the matrices $\M{A}$, $\M{B}$, and $\M{C}$ have more columns than rows, then the tensor gets smaller at each step.
While this standard method is work efficient, it involves communicating the intermediate tensors in each step, which are larger than the final output.

We propose that this communication can be \emph{deferred} until later steps, while still maintaining the correct output. We will call this proposed deferred-communication approach \emph{\dTTM}, noting that element $(i,j,k)$ of $\T{Y}$
can be written element-wise using~\cref{eqn:contractionelem} as
\begin{equation}
  \label{eqn:elemwisemttm}
  y_{ijk} = \sum_\mu \sum_\nu \sum_\xi x_{\mu \nu \xi}a_{\mu i}b_{\nu j}c_{\xi k}.
\end{equation}

While this approach isn't work optimal, the reduction in communication should yield useful applications, particularly for low-flop operations such as sketching. We propose to elaborate on this communication/computation tradeoff.

\section{Randomized Tensor Train Decomposition} \label{sec:tt} 
The exact `quasi-optimal' scheme for building a tensor train is TT-SVD~\cite{tensortrain}, where the first two steps for a 4-way tensor are illustrated in~\cref{fig:ttsequence}. As shown, This simply involves a series of reshape operations (permutations) followed by skinny SVDs. Provided that reshape operations can be efficiently performed in distributed memory (or avoided altogether), we can implement this method efficiently in distributed memory (using a `Gram' style SVD approach, as in~\cref{alg:sthosvd}). This would be the first know distribute implementation of a tensor train decomposition.

Furthermore, both common solution methods for computing a tensor train decomposition (SVD or optimization) may be able to efficiently leverage the previously discussed ideas from randomized SVD and randomized least squares, respectively. Thus we propose to explore the following questions:

\begin{itemize}
\item Can `sweeping' optimization algorithms for the tensor train be written in a way that can leverage randomized least squares, e.g., as described in~\cref{sec:sketching}?
\item Can the SVD method for forming the tensor train decomposition be expressed in a way that can be efficiently performed in distributed memory? E.g., starting from the distributed tensor representation outlined in~\cref{sec:tucker}.
\item Can ideas from randomized low-rank decompositions be applied efficiently to this formulation in distributed memory? E.g., using methods from~\cite{tropp2}. 
\end{itemize}

\begin{figure}
  \centering 
  \includegraphics[width=\linewidth]{thpropfigs/ttsequence}
  \caption{The creation of the first two `carriages' of a tensor train decomposition for a 4-way tensor.}
  \label{fig:ttsequence}
\end{figure}

\paragraph{Proposal}
We summarize our proposed contributions as follows:
\begin{itemize}
  \item Implement the first distributed-memory implementation of the tensor train decomposition.
  \item Analyze the communication and computation behavior of the `reshape-SVD' method in distributed memory.
  \item Apply methods from our randomized Tucker decomposition to the tensor train in distributed memory.
  \item Benchmark both methods on a supercomputer and analyze/establish tradeoffs.
\end{itemize}

% \section{Streaming Graph Partitioning} \label{sec:graph}
% \input{05-graph}
%
% \section{Timeline}

\bibliographystyle{siamplain}
\bibliography{bib,newbibs,tensbib}

\end{document}
