A standard method for computing the Tucker decomposition is the Higher-Order SVD 
(\hosvd)~\cite{Lathauwer00amultilinear}, a generalization of the singular value 
decomposition to higher-order datasets, i.e., $d$-way arrays with $d>2$.
% 
A direct approach to the \hosvd involves computing a series of SVDs of 
unfolded tensors to compute the orthogonal factor matrices whose 
columns approximately span the columns of the unfolded tensor (i.e., the 
mode-$k$ fiber space). We will refer to this computation as the Mode-wise Truncated Fiber Space Basis Computation (\MTFSBC).
Following this operation, the resulting factor matrices are applied to the input tensor to compress the original data.
%
The Tucker decomposition for $d=3$ is
illustrated in~\cref{fig:thirdordertucker}, where the original $d$-way tensor is 
denoted by $\T{X}$, the $d$ factor matrices are denoted as $\Mn{U}{k}$ for $k=1,2,3$,
and the $d$-way core tensor is denoted as $\T{G}$. 
If the data tensor has nearly low-rank structure, then the decomposed tensor uses significantly
less storage.
%

\begin{figure}[htbp]
  \centering
\begin{tikzpicture}[scale=0.5,namenode/.style={scale=.75}]
	\def\ix{3} %
	\def\iy{3} %
	\def\iz{2.5} %
	\def\corescale{1.75}
	\def\rx{\ix/\corescale}
	\def\ry{\iy/\corescale}
	\def\rz{\iz/\corescale}
	\coordinate (XFrontLowerLeft) at (0,0);
	\draw (XFrontLowerLeft) rectangle ++ (\ix,\iy); %
	\begin{scope}[shift={(XFrontLowerLeft)},canvas is zx plane at y=\iy,rotate=90]
	  \draw (0,0) rectangle ++ (\ix,\iz); %
	\end{scope}
	\begin{scope}[shift={(XFrontLowerLeft)},canvas is zy plane at x=\ix,rotate=90]
	  \draw (0,0) rectangle ++ (\iy,\iz); %
	\end{scope}
	\node[namenode] at ($(XFrontLowerLeft) + (0.5*\ix, 0.5*\iy)$)  {$\T{X}$};
	\coordinate (ApproxCtr) at ($(XFrontLowerLeft) + (\ix+0.4*\iz,0.75*\iy) + (0.75,0)$);

	\node[namenode] at (ApproxCtr) {$\approx$};
	\coordinate (U1LowerLeft) at ($(ApproxCtr) - (0,0.75*\iy) + (0.75,0)$);
	\draw (U1LowerLeft) rectangle ++ (\ry,\iy);

	\node[namenode] at ($(U1LowerLeft)+(0.5*\ry, 0.5*\iy)$)  {$\Mn{U}{1}$};
	\coordinate (GFrontLowerLeft) at ($(U1LowerLeft) + (\ry+0.5,1)$);
	\draw (GFrontLowerLeft) rectangle ++ (\rx,\ry);
	\begin{scope}[shift={(GFrontLowerLeft)},canvas is zx plane at y=\ry,rotate=90]
	  \draw (0,0) rectangle ++ (\rx,\rz);
	\end{scope}
	\begin{scope}[shift={(GFrontLowerLeft)},canvas is zy plane at x=\rx,rotate=90]
	  \draw (0,0) rectangle ++ (\ry,\rz);
	\end{scope}
	\node[namenode] at ($(GFrontLowerLeft)+(0.5*\rx,.5*\ry)$)  {$\T{G}$};
	\coordinate (U2LowerLeft) at ($(GFrontLowerLeft) + (\rx+\rz*0.4+0.5,0.5)$);
	\draw (U2LowerLeft) rectangle ++ (\ix,\rx); %

	\node[namenode] at ($(U2LowerLeft)+(0.5*\ix,0.5*\rx)$)  {$\Mn{U}{2}$};
	\coordinate (U3LowerLeft) at ($(GFrontLowerLeft) + (0.5,\ry+.8)$);
	\begin{scope}[shift={(U3LowerLeft)},canvas is zx plane at y=0,rotate=90]
	  \draw (0,0) rectangle ++ (\rz,\iz); %
	\end{scope}
	\node[namenode] at ($(U3LowerLeft)+(1.2,0.5)$) {$\Mn{U}{3}$};
\end{tikzpicture}
  \caption{Tucker decomposition of 3rd-order tensor ($d=3$).}
  \label{fig:tucker}
\end{figure}
%%% Local Variables:
%%% mode: latex
%%% TeX-master: t
%%% End:
 %%%%%% Tucker Figure %%%%%%
%
We consider a distributed-memory application for large dense tensors, as might arise in large-scale 
scientific simulations. For instance, we consider a five-way tensor representing three spatial dimensions, 
a time dimension, and a number of different variables. A user may wish to examine this data on a local 
machine with limited memory or to store many simulations in limited storage, and the \hosvd has been 
shown to be a highly scalable method for performing this compression.
The current state of the art for distributed \hosvd is TuckerMPI\footnote{\url{http://tensors.gitlab.io/TuckerMPI/}}, 
based on the work of Austin, Ballard, and Kolda~\cite{AuBaKo16}. The \MTFSBC kernel is performed with a distributed Gram matrix computation followed by a local eigensolve. The factors are then applied to the input tensor using a distributed tensor-matrix multiplication, forming the dense core tensor. The core tensor and factor matrices can be used to construct an approximation to the original data or any subset thereof.

In this work, we consider efficiently performing the \MTFSBC kernel in distributed memory by applying methods 
from randomized numerical linear algebra. We show how a series of local multiplications followed by a small 
global communication can result in a \emph{sketch} of the original tensor that fits in local memory. 
We see up to a 7X speedup in computing the \MTFSBC.
% with comparable error. 
Though randomized methods work with only a subset of the data, the errors are very close.
Finally, we apply this to the efficient computation of a Tucker decomposition for the compression of scientific data.
%The following is a more detailed breakdown of our contributions:
%Using this sketch we can, with the help of our theoretical results, compute an \hosvd that still satisfies (with high probability) the error tolerance. 
% \TODO{Include a plot of compression ratio / speed between TuckerMPI and sketching method.}
We summarize our contributions as follows:
\begin{itemize}
  \item We propose, implement, and develop a cost analysis for an efficient sketch-based \MTFSBC in 
  distributed memory, using randomization to reduce computation and communication. This method exploits the Kronecker structure of the sketching operation to reduce computation.
  \item We introduce a new parallel method for a sequence of tensor-times-matrix operations that trades extra computation for reduced communication and show how and when this can be used to further increase the efficiency of sketching. 
  \item We apply our kernel to the Tucker tensor decomposition, benchmark our method on large-scale scientific 
  data sets, and demonstrate scaling on synthetic data sets that improves upon the fastest known 
  traditional implementation.
  \item We provide theory to show how the error introduced through randomization can be compensated for.
  % No previous work has considered this issue, but it is key to preserving data accuracy.
\end{itemize}